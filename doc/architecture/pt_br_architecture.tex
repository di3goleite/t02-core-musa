%
% Portuguese-BR vertion
% 
\documentclass{report}

\usepackage{ipprocess}
% Use longtable if you want big tables to split over multiple pages.
% \usepackage{longtable}
\usepackage[utf8]{inputenc} 
\usepackage[brazil]{babel} % Uncomment for portuguese

\sloppy

\graphicspath{{./pictures/}} % Pictures dir
\makeindex
\begin{document}

\DocumentTitle{Documento de Arquitetura}
\Project{Core-MUSA}
\Organization{Universidade Estadual de Feira de Santana}
\Version{Build 2.0a}

\capa
\newpage
\newpage

%%%%%%%%%%%%%%%%%%%%%%%%%%%%%%%%%%%%%%%%%%%%%%%%%%
%% Revision History
%%%%%%%%%%%%%%%%%%%%%%%%%%%%%%%%%%%%%%%%%%%%%%%%%%
\chapter*{Histórico de Revisões}
  \vspace*{1cm}
  \begin{table}[ht]
    \centering
    \begin{tabular}[pos]{|m{2cm} | m{8cm} | m{4cm}|} 
      \hline
      \cellcolor[gray]{0.9}
      \textbf{Date} & \cellcolor[gray]{0.9}\textbf{Descrição} & \cellcolor[gray]{0.9}\textbf{Autor(s)}\\
      \hline 20/10/2014 &  Concepção do Documento & fmbboaventura \\ \hline
      		 23/10/2014 &  Revisão Inicial & jadsonfirmo \\ \hline
      		 29/10/2014 &  Adcionada Breve Descrição dos Componentes & fmbboaventura \\ \hline       
      		 29/10/2014 & Stakeholders & jadsonfirmo \\ \hline
    \end{tabular}
  \end{table}

% TOC instantiation
\tableofcontents

%%%%%%%%%%%%%%%%%%%%%%%%%%%%%%%%%%%%%%%%%%%%%%%%%%
%% Document main content
%%%%%%%%%%%%%%%%%%%%%%%%%%%%%%%%%%%%%%%%%%%%%%%%%%
\chapter{Introdução}
  
  \section{Propósito do Documento}
  Este documento descreve a arquitetura do projeto \ipPROCESSProject, incluindo especificações dos circuitos internos e máquinas de estados de cada componente. Ele também apresenta diagramas de classe, definições de entrada e saída e diagramas de temporização. O principal objetivo deste documento é definir as especificações do projeto \ipPROCESSProject\ e prover uma visão geral completa do mesmo.

  \noindent \textcolor{red}{\textit{Informações adicionais podem ser incluídas nesta seção. Entretanto, via de regra, ela não deve se extender por muitos parágrafos.}}

  \section{Stakeholders}
    \FloatBarrier
    \begin{table}[H] 
      \begin{center}
        \begin{tabular}[pos]{|m{6cm} | m{8cm}|} 
          \hline 
          \cellcolor[gray]{0.9}\textbf{Nome} & \cellcolor[gray]{0.9}\textbf{Papel/Responsabilidades} \\ \hline
          Diego Leite e Lucas Morais & Gerencia  \\ \hline
           Victor Figueiredo, Matheus Castro, Odivio Caio Santos e Kelvin Carmo & Desenvolvimento  \\ \hline
           Filipe Boaventura e Wagner Bittencourt & Implementação  \\ \hline
           Jadson Firmo & Análise e Refatoração  \\ \hline
        \end{tabular}
      \end{center}
    \end{table} 

\section{Visão Geral do Documento}

O presente documento é apresentado como segue:\\

  \begin{itemize}
   \item \textbf{Capítulo 2 --} Este capítulo apresenta uma visão geral da arquitetura, com foco em entrada e saída do sistema e arquitetura geral do mesmo.
   \item \textbf{Capítulo 3 --} Este capítulo apresenta a descrição detalhada da arquitetura bem como seus módulos e componentes.
  \end{itemize}


  % inicio das definições do documento
  \section{Definições}
    \FloatBarrier
    \begin{table}[H]
      \begin{center}
        \begin{tabular}[pos]{|m{5cm} | m{9cm}|} 
          \hline
          \cellcolor[gray]{0.9}\textbf{Termo} & \cellcolor[gray]{0.9}\textbf{Descrição} \\ \hline
                          &                       \\ \hline
        \end{tabular}
      \end{center}
    \end{table}  
  % fim

  % inicio da tabela de acronimos e abreviacoes do documento
  \section{Acrônimos e Abreviações}
    \FloatBarrier
    \begin{table}[H]
      \begin{center}
        \begin{tabular}[pos]{|m{2cm} | m{12cm}|} 
          \hline
          \cellcolor[gray]{0.9}\textbf{Sigla} & \cellcolor[gray]{0.9}\textbf{Descrição} \\ \hline
             PC       &  Contador de Programa (Program Counter)\\ \hline
             ULA      &  Unidade Lógica e Aritmética\\ \hline
             OPCODE   &  Código da Operação\\ \hline
        \end{tabular}
      \end{center}
    \end{table}  
  % fim

\chapter{Visão Geral da Arquitetura}

  \section{Descrição dos Componentes}
  A unidade de processamento a ser desenvolvida é composta a partir dos seguintes componentes:
  
  \begin{itemize}
    \item \textbf{PC --} Registrador que guarda o endereço da próxima instrução a ser executada.
    \item \textbf{Memória de Dados --} Parte da memória que armazena dados manipulados pelo programa.
    \item \textbf{Memória de Instrução --} Parte da memória que comtem o código do programa em linguagem de máquina.
    \item \textbf{ULA --} Implementa operações lógicas e aritméticas.
    \item \textbf{Unidade de Controle --} Decodifica a instrução e define sinais de controle para os demais componentes.
    \item \textbf{Banco de Registradores --} Contém os registradores de propósito geral do processador.
  \end{itemize}
  
  \section{Diagrama de Classe}

  \section{Definições de Entrada e Saída}

  \section{Datapath Interno}

% inicio das descrições de arquitetura para cada componente do sistema
\chapter{Descrição da Arquitetura}

  \section{Nome do Módulo}

    \subsection{Diagrama de Classe}

    \subsection{Definições de Entrada e Saída}

    \subsection{Datapath Interno}

    \subsection{Máquina de Estados}

    \subsection{Temporização}


% Optional bibliography section
% To use bibliograpy, first provide the ipprocess.bib file on the root folder.
% \bibliographystyle{ieeetr}
% \bibliography{ipprocess}

\end{document}
