%
% Portuguese-BR vertion
%
\documentclass{article}

\usepackage{ipprocess}
% Use longtable if you want big tables to split over multiple pages.
% \usepackage{longtable}
\usepackage[utf8]{inputenc}
\usepackage[brazil]{babel} % Uncomment for portuguese

\sloppy

\graphicspath{{./pictures/}} % Pictures dir
\makeindex
\begin{document}

\DocumentTitle{Especificação de Requisitos}
\Project{Core-MUSA}
\Organization{Universidade Estadual de Feira de Santana}
\Version{Build 0.1a}

\capa
\newpage

%%%%%%%%%%%%%%%%%%%%%%%%%%%%%%%%%%%%%%%%%%%%%%%%%%
%% Revision History
%%%%%%%%%%%%%%%%%%%%%%%%%%%%%%%%%%%%%%%%%%%%%%%%%%
\section*{\center Histórico de Revisões}
  \vspace*{1cm}
  \begin{table}[ht]
    \centering
    \begin{tabular}[pos]{|m{2cm} | m{7.2cm} | m{3.8cm}|} 
      \hline
      \cellcolor[gray]{0.9}
      \textbf{Data} & \cellcolor[gray]{0.9}\textbf{Descrição} & \cellcolor[gray]{0.9}\textbf{Autor(s)}\\ \hline
      \hline
      \small 27/09/2014 & \small Concepção do Documento & \small fmbboaventura \\ \hline     
      \small 12/08/2014 &
      \begin{small}
        \begin{itemize}
          \item Inclusão dos requisitos [FR2];
          \item Inclusão dos requisitos [NFR3];
          \item Inclusão das dependências [D4];
        \end{itemize}
      \end{small} & \small bezourokq, di3goleite, fmbboaventura, gordinh, jadsonfirmo, KelCarmo, mtcastro e wsbittencourt \\ \hline
      \small 30/09/2014 & \small Inclusão dos requisitos não funcionais & \small gordinh, Figueiredovr \\ \hline
      \small 30/09/2014 & \small Revisão do documento & \small gordinh \\ \hline
      \small 05/10/2014 & \small Atualização dos requisitos & \small gordinh \\ \hline
      \small 07/10/2014 & \small Atualização dos requisitos & \small gordinh, di3goleite \\ \hline
      \small 07/10/2014 & \small Atualização dos requisitos & \small Figueiredovr \\ \hline
      \small 07/10/2014 & \small Atualização dos requisitos não-funcionais & \small mtcastro \\ \hline
      \small 09/10/2014 & \small Revisão do documento & \small gordinh, di3goleite \\ \hline
      \small 09/10/2014 & \small Revisão do documento & \small Figueiredovr, mtcastro, di3goleite \\ \hline
      \small 20/10/2014 & \small Refatoração do documento & \small jadsonfirmo \\ \hline
      \small 23/10/2014 & \small Revisão & \small jadsonfirmo \\ \hline
      \small 28/10/2014 & \small Atualização de Requisitos & \small jadsonfirmo \\ \hline
      \small 30/10/2014 & \small Revisão Final & \small jadsonfirmo \\ \hline
    \end{tabular}
  \end{table}

\newpage

% TOC instantiation
\tableofcontents
\newpage

%%%%%%%%%%%%%%%%%%%%%%%%%%%%%%%%%%%%%%%%%%%%%%%%%%
%% Document main content
%%%%%%%%%%%%%%%%%%%%%%%%%%%%%%%%%%%%%%%%%%%%%%%%%%
\section{Introdução}
	Este documento apresenta de forma detalhada os requisitos do sistema, sendo estes dividos em funcionais e não funcionais. Também apresenta as dependências para o funcionamento do sistema.

\subsection{Visão Geral do Documento}
  \begin{itemize}
   \item \textbf{Requisitos funcionais -} Lista de todos os requisitos funcionais.
   \item \textbf{Requisitos não funcionais -} Lista de todos os requisitos não funcionais.
   \item \textbf{Dependências -} Conjunto de dependências de IP-cores previstos.
   \item \textbf{Notas -} apresenta a lista de notas apresentadas ao longo do documento.
   \item \textbf{Referências -} Lista de todos os textos referenciados nesse documento.
  \end{itemize}

  % inicio das definições do documento
  \subsection{Definições}
    \FloatBarrier
    \begin{table}[H]
      \begin{center}
        \begin{tabular}[pos]{|m{5cm} | m{9cm}|}
          \hline
          \cellcolor[gray]{0.9}\textbf{Termo} & \cellcolor[gray]{0.9}\textbf{Descrição} \\ \hline
          Requisitos Funcionais & Requisitos de hardware que compõem os módulos, descrevendo as ações que o mesmo deve estar apto a executar. Estas informações são capturadas a partir do desenvolvimento dos casos de uso, que documentam as entradas, os processos e as saídas geradas.  \\ \hline
          Requisitos Não Funcionais & Requisitos de hardware que compõem os módulos, representando as características que o mesmo deve ter, ou restrições que o mesmo deve operar. Estas características referem-se técnicas, algoritmos, tecnologias e especificidades do Sistema como um todo.  \\ \hline
          Dependências & Requisitos de reuso de IP-cores, descrevendo as funções que cada um deve exercer. \\ \hline
        \end{tabular}
      \end{center}
    \end{table}
  % fim

  % inicio da tabela de acronimos e abreviacoes do documento
  \subsection{Acrônimos e Abreviações}
    \FloatBarrier
    \begin{table}[H]
      \begin{center}
        \begin{tabular}[pos]{|m{2cm} | m{12cm}|}
          \hline
          \cellcolor[gray]{0.9}\textbf{Sigla} & \cellcolor[gray]{0.9}\textbf{Descrição} \\ \hline
          FR      & Requisito Funcional  \\ \hline
          NFR     & Requisito Não Funcional  \\ \hline
          D       & Dependência  \\ \hline
          PC      & Program Counter (Contador de Programa)  \\ \hline
        \end{tabular}
      \end{center}
    \end{table}
  % fim

  % inicio da descriao de prioridades de requisitos
  \subsection{Prioridades dos Requisitos}
    \FloatBarrier
    \begin{table}[H]
      \begin{center}
        \begin{tabular}[pos]{|m{2cm} | m{12cm}|}
          \hline
          \cellcolor[gray]{0.9}\textbf{Prioridade} & \cellcolor[gray]{0.9}\textbf{Característica} \\ \hline
          Importante      & Requisito sem o qual o sistema funciona, porém não como deveria.  \\ \hline
          Essencial       & Requisito que deve ser implementado para que o sistema funcione.  \\ \hline
          Desejável       & Requisito que não compromete o funcionamento do sistema.  \\ \hline
        \end{tabular}
      \end{center}
    \end{table}
  % fim

  % inicio dos requisitos funcionais
  \section{Requisitos Funcionais}
  \subsection{Instruções de Movimentação de Dados}
    \begin{functional}
     % \requirement{name}{description}{priority}
     \requirement
      {Instrução LW}
      {O processador deve ser capaz de ler valores da memória de dados.
       A instrução LW está compreendida da seguinte forma:\\
       \begin{itemize}
        \item Registrador de Destino (RD) - registrador onde será armazenado o valor que será carregado da memória.
        \item Registrador Fonte (RS) - registrador fonte onde contém o dado a ser lido.
        \item Deslocamento (I16) - endereço, de 16 bits, o qual haverá o deslocamento de bits a partir do RS.
       \end{itemize}
       }
      {Essencial}

     \requirement
      {Instrução SW}
      {O processador deve ser capaz de inserir valores na memória de dados.
      A instrução SW está compreendida da seguinte forma:\\
       \begin{itemize}
        \item Registrador Fonte (RS) - registrador fonte onde contém o dado a ser salvo.
        \item Registrador de Destino (RD) - registrador onde será armazenado o valor proveniente do RS.
        \item Deslocamento (I16) - endereço, de 16 bits, o qual haverá o deslocamento de bits a partir do RD.
       \end{itemize}
       }
      {Essencial}
    \end{functional}

    \subsection{Computacionais}
    \begin{functional}

     \requirement
      {Instrução ADD}
      {O processador deve ser capaz de realizar a soma de dois valores, levando em consideração o sinal.
      A instrução ADD está compreendida da seguinte forma:\\
       \begin{itemize}
        \item Registrador Fonte 1 (RS1) - registrador fonte representando o operando1.
        \item Registrador Fonte 2 (RS2) - registrador fonte representando o operando2.
        \item Registrador de Destino (RD) - registrador onde será armazenado o valor da soma entre o operando1 e operando2 (RS1+RS2).
       \end{itemize}
       }
      {Essencial}

     \requirement
      {Instrução SUB}
      {O processador deve ser capaz de realizar a subtração de dois valores, levando em consideração o sinal.
      A instrução SUB está compreendida da seguinte forma:\\
       \begin{itemize}
        \item Registrador Fonte 1 (RS1) - registrador fonte representando o operando1.
        \item Registrador Fonte 2 (RS2) - registrador fonte representando o operando2.
        \item Registrador de Destino (RD) - registrador onde será armazenado o valor da subtração entre o operando1 e operando2 (RS1-RS2).
       \end{itemize}
       }
      {Essencial}

      \requirement
      {Instrução MUL}
      {O processador deve ser capaz de realizar a multiplicação de dois valores, levando em consideração o sinal.
      A instrução MUL está compreendida da seguinte forma:\\
       \begin{itemize}
        \item Registrador Fonte 1 (RS1) - registrador fonte representando o operando1.
        \item Registrador Fonte 2 (RS2) - registrador fonte representando o operando2.
        \item Registrador de Destino (RD) - registrador onde será armazenado o valor da multiplicação entre o operando1 e operando2 (RS1*RS2).
       \end{itemize}
       }
      {Essencial}

       \requirement
      {Instrução DIV}
      {O processador deve ser capaz de realizar a divisão de dois valores, levando em consideração o sinal.
      A instrução DIV está compreendida da seguinte forma:\\
       \begin{itemize}
        \item Registrador Fonte 1 (RS1) - registrador fonte representando o operando1.
        \item Registrador Fonte 2 (RS2) - registrador fonte representando o operando2.
        \item Registrador de Destino (RD) - registrador onde será armazenado o valor da divisão entre o operando1 e operando2 (RS1/RS2).
       \end{itemize}
       }
      {Essencial}
      
      \requirement
      {Instrução ADDi}
      {O processador deve ser capaz de realizar a soma de dois valores (levando em consideração o sinal), de modo que um desses valores é um valor imediato. A instrução ADDi está compreendida da seguinte forma:\\
       \begin{itemize}
        \item Registrador Fonte 1 (RS1) - registrador fonte representando o operando1.
        \item Registrador Fonte 2 (RS2) - registrador fonte representando o valor imediato.
        \item Registrador de Destino (RD) - registrador onde será armazenado o valor da soma entre o operando1 e o imediato (RS1+RS2).
       \end{itemize}
       }
      {Essencial}
      
      \requirement
      {Instrução SUBi}
      {O processador deve ser capaz de realizar a subtração de dois valores (levando em consideração o sinal), de modo que um desses valores é um valor imediato. A instrução SUBi está compreendida da seguinte forma:\\
       \begin{itemize}
        \item Registrador Fonte 1 (RS1) - registrador fonte representando o operando1.
        \item Registrador Fonte 2 (RS2) - registrador fonte representando o valor imediato.
        \item Registrador de Destino (RD) - registrador onde será armazenado o valor da subtração entre o operando1 e o imediato (RS1-RS2).
       \end{itemize}
       }
      {Essencial}

     \requirement
      {Instrução CMP}
      {O processador deve ser capaz de comparar dois registradores e ativar ou desativar uma flag para sinalizar igualdade.
      A instrução CMP está compreendida da seguinte forma:\\
      \begin{itemize}
       \item Registrador 1 (RS) - registrador representando um valor que será comparado com o RD.
       \item Registrador 2 (RD) - registrador representando um valor que será comparado com o RS.
      \end{itemize}
      }
      {Essencial}

      \requirement
      {Instrução AND}
      {O processador deve ser capaz de realizar a operação lógica AND, bit a bit, de dois valores.
      A instrução AND está compreendida da seguinte forma:\\
       \begin{itemize}
        \item Registrador Fonte 1 (RS1) - registrador fonte representando o operando1.
        \item Registrador Fonte 2 (RS2) - registrador fonte representando o operando2.
        \item Registrador de Destino (RD) - registrador onde será armazenado o valor da operação AND entre o operando1 e operando2 (RS1\&\&RS2).
       \end{itemize}
       }
      {Essencial}

      \requirement
      {Instrução OR}
      {O processador deve ser capaz de realizar a operação lógica OR, bit a bit, de dois valores.
      A instrução OR está compreendida da seguinte forma:\\
       \begin{itemize}
        \item Registrador Fonte 1 (RS1) - registrador fonte representando o operando1.
        \item Registrador Fonte 2 (RS2) - registrador fonte representando o operando2.
        \item Registrador de Destino (RD) - registrador onde será armazenado o valor da operação OR entre o operando1 e operando2 (RS1||RS2).
       \end{itemize}
       }
      {Essencial}

      \requirement
      {Instrução NOT}
      {O processador deve ser capaz de realizar a operação lógica NOT, de inversão, bit a bit.
      A instrução NOT está compreendida da seguinte forma:\\
       \begin{itemize}
         \item Registrador de Destino (RD) - registrador onde contém o valor a ser negado e onde será armazenado este valor.
         \end{itemize}
         }
      {Essencial}
      
      \requirement
      {Instrução ANDi}
      {O processador deve ser capaz de realizar a operação lógica AND, bit a bit, de dois valores, de modo que um desses valores é um valor imediato.
      A instrução ANDi está compreendida da seguinte forma:\\
       \begin{itemize}
        \item Registrador Fonte 1 (RS1) - registrador fonte representando o operando1.
        \item Registrador Fonte 2 (RS2) - registrador fonte representando o valor imediato.
        \item Registrador de Destino (RD) - registrador onde será armazenado o valor da operação AND entre o operando1 e o imediato (RS1\&\&RS2).
       \end{itemize}
       }
      {Essencial}
      
      \requirement
      {Instrução ORi}
      {O processador deve ser capaz de realizar a operação lógica OR, bit a bit, de dois valores, de modo que um desses valores é um valor imediato.
      A instrução ORi está compreendida da seguinte forma:\\
       \begin{itemize}
        \item Registrador Fonte 1 (RS1) - registrador fonte representando o operando1.
        \item Registrador Fonte 2 (RS2) - registrador fonte representando o valor imediato.
        \item Registrador de Destino (RD) - registrador onde será armazenado o valor da operação OR entre o operando1 e o imediato (RS1||RS2).
       \end{itemize}
       }
      {Essencial}

      \requirement
      {Tamanho da palavra de uma instrução de 32 bits}
      {O Tamanho de uma palavra de instrução que é compatível com o processador é de 32 bits.}
      {Essencial}
\end{functional}
    \subsection{Instruções de Desvio}

    \begin{functional}
      \requirement
      {Instrução JR}
      {O processador deve ser capaz de desviar um programa em execução para um endereço de destino.
      A instrução JR está compreendida da seguinte forma:\\
       \begin{itemize}
         \item Registrador Endereço (R) - registrador onde contém o endereço para onde o programa deverá ir.
         \end{itemize}
         }
      {Essencial}

      \requirement
      {Instrução JPC}
      {O processador deve ser capaz de desviar um programa em execução para um endereço relativo ao PC.
      A instrução JPC está compreendida da seguinte forma:\\
       \begin{itemize}
         \item Registrador de Endereço (RD) - registrador onde contém o valor, de 28 bits, relativo ao PC, para onde o programa deverá ir.
         \end{itemize}
         }
      {Essencial}

      \requirement
      {Instrução BRFL}
      {O processador deve ser capaz de desviar um programa em execução para um endereço de destino atendendo a uma condição de flag.
      A instrução BRFL está compreendida da seguinte forma:\\
       \begin{itemize}
         \item Registrador (R) - registrador que contém o endereço para salto.
         \item Contante (CST) - constante a qual será comparada com uma flag, de modo que se o resultado da comparação tiver resultado verdadeiro, o programa salta para o endereço armazenado no registrador (R).
         \end{itemize}
         }
      {Essencial}

      \requirement
      {Instrução CALL}
      {O processador deve ser capaz de desviar um programa em execução para uma sub-rotina.
      A instrução CALL está compreendida da seguinte forma:\\
       \begin{itemize}
         \item Registrador de Destino (RD) - registrador onde contém o valor para onde o programa deverá ir. Essa instrução salva o endereço atual do pc, armazenado-o na pilha.
        \end{itemize}
        }
      {Essencial}

      \requirement
      {Instrução RET}
      {O processador deve ser capaz de retornar de uma sub-rotina.
      A instrução RET está compreendida da seguinte forma:\\
      \begin{itemize}
         \item A instrução deverá acessar a pilha procurando pelo endereço de retorno. Ao acessar a pilha, que contém os endereços do PC salvos, o programa deve ir para o endereço que está no topo da pilha.
        \end{itemize}
        }
      {Essencial}

    \end{functional}

    \subsection{Outras Instruções}

    \begin{functional}

      \requirement
      {Instrução NOP}
      {O processador deve ser capaz de não realizar operações durante os 5 ciclos de clock.}
      {Essencial}

      \requirement
      {Instrução HALT}
      {O processador deve ser capaz de parar a execução de um programa.}
      {Essencial}

    \end{functional}

    \subsection{Flags}

    \begin{functional}

      \requirement{Overflow/Underflow}
      {O processador deve ser capaz de avisar que houve um erro (ou um "estouro") na operação aritimética através da flag de overflow/underflow.}
      {Essencial}

      \requirement{Equals}
      {Esta Flag deve ser utilizada como resultado da instrução CMP, deve constar verdadeiro quando as duas palavras comparadas forem iguais.}
      {Essencial}

      \requirement{Above}
      {Esta Flag deve ser utilizada como resultado da instrução CMP, deve constar verdadeiro quando uma palavra é maior que a outra.}
      {Essencial}

      \requirement{Error}
      {Deve indicar a ocorrência de uma operação não válida, como por exemplo divisão por zero.}
      {Essencial}

     \end{functional}

     \subsection{Modos de Endereçamento}

    \begin{functional}

      \requirement{Imediato}
      {O processador deve aceitar instruções onde o dado já é passado explicitamente na instrução.}
      {Essencial}

       \requirement{Deslocamento de Base}
      {O processador deve aceitar instruções onde os operandos contenham o endereço da base e o valor do deslocamento.}
      {Essencial}

      \requirement{Por Registrador}
      {O processador deve aceitar instruções onde o endereço do registrador seja passado como parâmetro.}
      {Essencial}

     \end{functional}


\section{Requisitos não Funcionais}
Esta seção apresenta a lista de Requisitos não Funcionais do projeto.

  \begin{nonfunctional}

    \requirement
    {Ferramenta para simulação de testes}
    {Será utilizado o programa ModelSim®-Altera Web Edition, para fazer a simulação dos módulos e testes dos mesmos.}
    {Importante}

    \requirement
    {Ferramenta para prototipação}
    {Será utilizada a Plataforma de Desenvolvimento FPGA  Terasic ALTERA Cyclone III (EP3C25F324) para a execução do protótipo.}
    {Importante}

    \requirement
    {Linguagem de Descrição}
    {Tanto o projeto quanto os testes serão descritos usando Verilog-HDL.}
    {Desejável}

    \requirement
    {Plano de Teste}
    {Será desenvolvido um conjunto de programa de teste para cado bloco implementado no projeto. }
    {Desejável}

    \requirement{Organização dos Dados}
    {Os bytes são numerados da esquerda para a direita: formato Big Endian.}
    {Essencial}

    \requirement
    {Ferramenta para programar o processador}
    {Será utilizado o software Quartus para descarregar os programas que serão executados no processador.}
    {Importante}


    \requirement{Tempo de Operação}
    {Toda instrução deve ser executada em exatamente cinco ciclos de clock.}
    {Essencial}

  \end{nonfunctional}

\section{Dependências}
  % Esta seção apresenta uma lista dos IP-cores disponíveis % para reuso e que devem ser adotados no desenvolvimento % deste projeto.

  \begin{dependencies}
    \dependency{ULA}{Módulo da ULA implementado no projeto \textit{Warmup}, contando com algumas adaptações e incremento de operações.}
\end{dependencies}

% Optional bibliography section
% To use bibliograpy, first provide the ipprocess.bib file on the root folder.
% \bibliographystyle{ieeetr}
% \bibliography{ipprocess}

\end{document}